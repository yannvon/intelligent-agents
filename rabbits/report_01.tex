\documentclass[11pt]{article}

\usepackage{amsmath}
\usepackage{textcomp}
\usepackage{listings}
\lstset{
  basicstyle={\ttfamily},
  language=Java,
  tabsize=2,
  showstringspaces=false
  %frame=tb,
  aboveskip=5mm,
  %belowskip=3mm,
  columns=fixed,
  %xleftmargin=5mm,
  %framexleftmargin=4mm,
  %numbers=left,
  %numbersep=3pt,
  %numberstyle=\small\color{darkgray},
  stringstyle=\em,
  %frame=single,
  breaklines=true,
  breakatwhitespace=true
}
\usepackage[top=0.8in, bottom=0.8in, left=0.8in, right=0.8in]{geometry}
% Add other packages here %



% Put your group number and names in the author field %
\title{\bf Excercise 1.\\ Implementing a first Application in RePast: A Rabbits Grass Simulation.}
\author{Group \textnumero46: Yann Vonlanthen, Lo\"ic Vandenberghe}

\begin{document}
\maketitle

\section{Implementation}

\subsection{Assumptions}
\begin{itemize}
\item The parameters \lstinline!minEnergy! and \lstinline!maxEnergy! decide the maximum and minimum energy that a rabbit can start with. Once a rabbit is created, its energy is chosen randomly in $[minEnergy, maxEnergy)$. If the minimum is bigger than the maximum, the value is chosen in $[1, maxEnergy)$ instead.
\item The parameters \lstinline!minGrassCal! and \lstinline!maxGrassCal! decide the minimum and maximum level of calories of a grass. For each grass created, its calories level is set randomly between thoses values. Once a grass is eaten by a rabbit, the rabbit gain the calories as energy.
 \item Rabbits that dies are removed at the beginning of the next tick. Which means if the rabbit consume its last point of energy, its corpse stays on the field for one tick and then gets removed next tick before any rabbit takes a step.
 \item If a rabbit gets enough energy to reproduce, its own energy is reset between \lstinline!minEnergy! and \lstinline!maxEnergy! and a new rabbit is added in a pending queue, the new rabbit is added only at the beginning of the next step.
\end{itemize}


\subsection{Implementation Remarks}
% Provide important details about your implementation, such as handling of boundary conditions %

\section{Results}
% In this section, you study and describe how different variables (e.g. birth threshold, grass growth rate etc.) or combinations of variables influence the results. Different experiments with diffrent settings are described below with your observations and analysis

\subsection{Experiment 1}

\subsubsection{Setting}

\subsubsection{Observations}
% Elaborate on the observed results %

\subsection{Experiment 2}

\subsubsection{Setting}

\subsubsection{Observations}
% Elaborate on the observed results %

\vdots

\subsection{Experiment n}

\subsubsection{Setting}

\subsubsection{Observations}
% Elaborate on the observed results %

\end{document}
